\chapter{Metodologia}
\label{chap:metodologia}
%--------- NEW SECTION ----------------------
A pesquisa do estudo do estado da arte desenolvido neste documento foi elaborada principalmente a partir do método BILI, que permite realizar uma pesquisa bibliográfica em um banco de dados de artigos científicos, publicações em periódicos, livros e outras fontes de conhecimento científicos, fazendo o levantamento das publicações e dos autores mais impactantes na área pesquisada. Também foram realizadas pesquisas para avaliar as soluções encontradas já no mercado.

O método BILI é dívidido em quatro ciclos que acontecem em sequência. Eles são chamados de ciclo ingênuo, ciclo otimizada, ciclo de impacto e ciclo de produção.

% BIBLIOMETRIX


\begin{figure} [h!]	
  \centering
  \caption{Método BILI}
  \includegraphics[width=1\textwidth]{Figures/bili.png}
  \caption*{Fonte: Autoria própria.}
  \label{fig:QFD}
\end{figure}

\section{Ciclo Ingênuo}
No primeiro ciclo do método BILI é feita uma busca das pesquisas desenvolvidas na área estudada em um banco de dados de documentos acadêmicos, através de palavras-chave que tenham conexão o tema proposto. O banco de dados escolhido para aplicar o método foi o Scopus.

O resultado da busca no banco de dados com as palavras-chave é um arquivo .bib, que contém o nome dos documentos encontrados, palavras-chave, ano de publicação, nome dos autores, DOI, resumo, entre outros.

Em seguida o arquivo .bib gerado no banco de dados é aberto no Bibliometrix, que é um pacote desenvolvido para R que permite uma clara visualização das informações levantadas e também realiza análises dos dados obtido. 

É então realizada a análise da rede de cocitação, que é um gráfico que relaciona os autores, mostrando a proximidade da pesquisa realizada por eles através de citações realizadas nas pesquisas. O resultado dessa análise é positivo se a rede de cocitação estiver coesa, com todos elementos conectados.

É avaliada também a produção científica anual, que é um gráfico que mostra quantas pesquisas foram realizadas na área pesquisada ao longo do tempo. O resultado desse gráfico é coerente quando se tem um crescimento positivo do número de pesquisas realizadas com o passar do tempo.

São avaliados por último os gráficos de histograph e wordcloud, que dão idéia das pesquisas mais importantes realizadas ao longo do tempo e das palavras chaves mais utilizadas, respectivamente.

Se a pesquisa não apresentar resultados coerentes, é realizada uma nova pesquisa com palavras-chave diferentes. Caso seja coerente, o próximo passo é fazer um refinamento das palavras-chave através do litsearchr, que é um pacote desenvolvido para R, que através do arquivo .bib obtido, fornece as palavras-chave mais impactantes dos dados.

\section{Ciclo Otimizado}
Com as palavras-chave otimizadas obtidas através do litsearchr, é feita uma nova busca no banco de dados acadêmico escolhido, no caso o Scopus, para obter um novo arquivo .bib, da mesma forma que foi obtido no primeiro ciclo.

Os gráfico de rede de cocitação, produção científica anual, wordcloud e histograph são analisados novamente, para avaliar a coerência. Caso verificada a coerência do resultado, é realizada uma filtragem dos resultados através do revtools. O revtools é um pacote desenvolvido também para R onde é possível realizar a leitura do resumos das pesquisas e é possível manter a pesquisa caso ela seja útil ou excluir caso ela não sirva.

O resultado do revtools é um arquivo .csv contendo apenas os documentos que são úteis para a pesquisa. O arquivo então é convertido novamente para o formato .bib e então passado para o próximo ciclo.

\section{Ciclo de Impacto}
O arquivo .bib gerado no ciclo anterior é novamente aberto no Bibliometrix. Nessa etapa é avaliados o gráfico de author impact by total citation e levantados de três a cinco autores com maior impacto apresentados no gráfico.

\section{Ciclo de Produção}
Por último são coletados os artigos dos autores selecionados no banco de dados e é feita a leitura completa dos artigos. 

É feito o upload dos artigos e leitura na plataforma Mendeley. Nesse aplicativo é possível compartilhar os artigos com grupos de estudo e também fazer anotações e grifar trechos importantes. 

A partir dos conhecimentos adquiridos é feito um mapa conceitual que relaciona os principais conceitos apresentados para servir de base para o desenvolvimento desse documento.

%----------------------------------------------------------

%--------- NEW SECTION ----------------------


%---------------picture------------------------------------
% \begin{figure}
%     \centering
%     \subfigure[Figure A]{\label{fig:a}\includegraphics[width=60mm]{./lq}}
%     \subfigure[Figure B]{\label{fig:b}\includegraphics[width=60mm]{./lq}}
%     \subfigure[Figure C]{\label{fig:c}\includegraphics[width=\textwidth]{./lq}}
%     \caption{Three simple graphs}
%     \label{fig:three graphs}
% \end{figure}
%----------------------------------------------------------

% \begin{figure}
%     \centering
%     \begin{subfigure}[b]{0.3\textwidth}
%         \centering
%         \includegraphics[width=\textwidth]{./lq}
%         \caption{$y=x$}
%         \label{fig:y equals x}
%     \end{subfigure}
%     \hfill
%     \begin{subfigure}[b]{0.3\textwidth}
%         \centering
%         \includegraphics[width=\textwidth]{./lq}
%         \caption{$y=3sinx$}
%         \label{fig:three sin x}
%     \end{subfigure}
%     \hfill
%     \begin{subfigure}[b]{0.3\textwidth}
%         \centering
%         \includegraphics[width=\textwidth]{./lq}
%         \caption{$y=5/x$}
%         \label{fig:five over x}
%     \end{subfigure}
%        \caption{Three simple graphs}
%        \label{fig:three graphs}
% \end{figure}


%--------- NEW SECTION ----------------------
% \section{Assunto 2}
% \label{sec:ass2}
% flkjasdlkfjasdlkfjs

% \begin{table}[h]
%     \begin{subtable}[h]{0.45\textwidth}
%         \centering
%         \begin{tabular}{l | l | l}
%         Day & Max Temp & Min Temp \\
%         \hline \hline
%         Mon & 20 & 13\\
%         Tue & 22 & 14\\
%         Wed & 23 & 12\\
%         Thurs & 25 & 13\\
%         Fri & 18 & 7\\
%         Sat & 15 & 13\\
%         Sun & 20 & 13
%        \end{tabular}
%        \caption{First Week}
%        \label{tab:week1}
%     \end{subtable}
%     \hfill
%     \begin{subtable}[h]{0.45\textwidth}
%         \centering
%         \begin{tabular}{l | l | l}
%         Day & Max Temp & Min Temp \\
%         \hline \hline
%         Mon & 17 & 11\\
%         Tue & 16 & 10\\
%         Wed & 14 & 8\\
%         Thurs & 12 & 5\\
%         Fri & 15 & 7\\
%         Sat & 16 & 12\\
%         Sun & 15 & 9
%         \end{tabular}
%         \caption{Second Week}
%         \label{tab:week2}
%      \end{subtable}
%      \caption{Max and min temps recorded in the first two weeks of July}
%      \label{tab:temps}
% \end{table}