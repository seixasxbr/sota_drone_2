\chapter{Introdução}
\label{chap:intro}

Este documento aborda o estudo do estado da arte de veículos aéreos não tripulados (VANTs) do tipo quadrotor, apresentando os principais estudos acadêmicos, técnicas e modelos dos últimos anos para embasar o desenvolvimento de um projeto envolvendo a concepção de um drone desse tipo. 

%--------- NEW SECTION ----------------------
\section{Objetivos}
\label{sec:obj}

Este estudo foi realizado para dar suporte no denvolvimento de um quadrotor autônomo. Com objetivo de trazer conhecimento das melhores técnicas que vem sendo utilizadas em áreas como navegação, controle e localização e mapeamento simultâneos (SLAM), assim como os melhores modelos e arquiteturas para conceber um veículo eficiente.

%--------- NEW SECTION ----------------------
\section{Justificativa}
\label{sec:justi}

Os VANTs tem sido cada vez mais utilizados para fins civís e militares. Tarefas que envolvem risco podem ser facilmente executadas por esse tipo de aeronave sem expor o piloto aos perigos associados a essa missão. Os drones tem sido utilizados em áreas como cinematografia, cartografia, vigilância, entrega de encomendas, mapeamento, entre outras. Devido a isso surge a importância de estudar essas aeronaves que apresentam alguns desafios a serem enfrentados como autonomia de vôo, localização e controle.

%--------- NEW SECTION ----------------------
\section{Organização do documento}
\label{section:organizacao}

Este documento apresenta $5$ capítulos e está estruturado da seguinte forma:

\begin{itemize}

  \item \textbf{Capítulo \ref{chap:intro} - Introdução}: Contextualiza o âmbito, no qual a pesquisa proposta está inserida. Apresenta, portanto, a definição do problema, objetivos e justificativas da pesquisa e como este \thetypeworkthree está estruturado;
  \item \textbf{Capítulo \ref{chap:fundteor} - Fundamentação Teórica}: XXX;
  \item \textbf{Capítulo \ref{chap:metod} - Materiais e Métodos}: XXX;
  \item \textbf{Capítulo \ref{chap:result} - Resultados}: XXX;
  \item \textbf{Capítulo \ref{chap:conc} - Conclusão}: Apresenta as conclusóes, contribuições e algumas sugestões de atividades de pesquisa a serem desenvolvidas no futuro.

\end{itemize}
