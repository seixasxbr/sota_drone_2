
\chapter{Introdução}
\label{chap:intro}

Este documento aborda o estudo do estado da arte de veículos aéreos não tripulados (VANTs) do tipo quadrotor, que são aeronaves de asas rotativas com quatro propulsores que podem se deslocar em todas as direções no espaço tridimensional. Esses veículos são comumentes chamados de \textit{drones}, que em inglês significa zangão ou zumbido, pelo barulho gerado pelos seus rotores em sua operação.

Os quadrotores são veículos muito ágeis, com grande manobrabilidade e \textit{payload}, além de possuírem a habilidade de realizar vôos estacionários e também pouso e decolagem vertical. Esses atributos tornam essas aeronaves muito úteis em diversas aplicações. Entretanto existem desafios a serem enfrentados na concepção de uma plataforma desse tipo, como  escolha de componentes, controle, localização, tempo de vôo e planejamento de trajetória.

Esta pesquisa apresenta os principais estudos acadêmicos realizados sobre este tipo de aeronave, principais autores, técnicas mais utilizadas aplicadas em suas funcionalidades e modelos já desenvolvidos, para embasar o desenvolvimento de um projeto de criação de um veículo desse tipo.

%--------- NEW SECTION ----------------------
\section{Objetivos}
\label{sec:obj}

Este estudo foi realizado para dar suporte no desenvolvimento de um quadrotor autônomo com capacidade de realizar pouso em uma plataforma móvel. Trazendo conhecimento das melhores técnicas que vem sendo utilizadas em áreas como navegação, controle e localização e mapeamento simultâneos (SLAM), assim como os melhores modelos, arquiteturas para conceber um veículo eficiente e principais componentes.

%--------- NEW SECTION ----------------------
\section{Justificativa}
\label{sec:justi}

Os VANTs tem sido cada vez mais utilizados para fins civis e militares. Tarefas que envolvem risco podem ser facilmente executadas por esse tipo de aeronave sem expor o piloto aos perigos associados a essa missão. Os drones tem sido utilizados em áreas como cinematografia, cartografia, vigilância, entrega de encomendas, mapeamento, entre outras. Devido a isso surge a importância de estudar essas aeronaves que apresentam alguns desafios a serem enfrentados como autonomia de vôo, localização e controle.

%--------- NEW SECTION ----------------------
\section{Organização do documento}
\label{section:organizacao}

Este documento apresenta $5$ capítulos e está estruturado da seguinte forma:

\begin{itemize}

  \item \textbf{Capítulo \ref{chap:intro} - Introdução}: 
  Apresenta o estudo que é desenvolvido neste documento, contextualizando a pesquisa, trazendo o objetivo principal, a justificativa e a organização do documento.
  \item \textbf{Capítulo \ref{chap:ambiente} - Ambiente de Desenvolvimento}: Apresenta os possíveis ambientes de atuação desse tipo de plataforma, a situação atual do desenvolvimento e as aplicações de mercado;
  \item \textbf{Capítulo \ref{chap:metodologia} - Metodologia}: Apresenta a metodologia utilizada para desenvolver esta pesquisa;
  \item \textbf{Capítulo \ref{chap:metod} - Estudo do Estado da Arte}: Apresenta os conceitos básicos de quadrotores, as classificações, principais componentes no desenvolvimento de uma plataforma desse tipo, principais funcionalidades, a revisão bibliográfica e o mapa conceitual;
  \item \textbf{Capítulo \ref{chap:conc} - Conclusão}: Apresenta as conclusões, resultados obtidos e contribuições desta pesquisa.

\end{itemize}
