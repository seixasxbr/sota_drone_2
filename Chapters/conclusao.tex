\chapter{Conclusão}
\label{chap:conc}

Nesse documento foi feito o estudo do estado da arte de veículos aéreos não tripulados do tipo quadrotor através do método BiLi. Primeiramente foi feito um estudo do ambiente de desenvolvimento que os quadrotores se encontram, com seu ambiente de aplicação, estado do desenvolvimento e mercado de atuação. O método BiLi foi importante para trazer os autores mais impactantes e áreas mais pesquisadas sobre o tema, sendo feitas outras buscas no Scopus mais direcionadas para cada tema para complementar o acervo bibliográfico. Foi feito um estudo das possíveis configurações de quadrotores e suas classificações. Foi feito um estudo dos principais componentes e opções disponíveis no mercado para integrar essa plataforma. E por último o estudo das principais funcionalidades envolvidas nos projetos de quadrotor, como localização, controle, planejamento de trajetória e energia. 


\section{Considerações finais}
\label{sec:consid}

Brevemente comentada no texto acima, nesta se\c{c}\~ao o
pesquisador (i.e. autor principal do trabalho cient\'ifico) deve
apresentar sua opini\~ao com respeito \`a pesquisa e suas
implica\c{c}\~oes. Descrever os impactos (i.e.
tecnol\'ogicos,sociais, econ\^omicos, culturais, ambientais,
políticos, etc.) que a pesquisa causa. N\~ao se recomenda citar
outros autores.

