\chapter{Conclusão}
\label{chap:conc}

Nesse documento foi feito o estudo do estado da arte de veículos aéreos não tripulados do tipo quadrotor através do método BiLi. Primeiramente foi feito um estudo do ambiente de desenvolvimento que os quadrotores se encontram, com seu ambiente de aplicação, estado do desenvolvimento e mercado de atuação. O método BiLi foi importante para trazer os autores mais impactantes e áreas mais pesquisadas sobre o tema, sendo feitas outras buscas no Scopus mais direcionadas para cada tema para complementar o acervo bibliográfico. Foi feito um estudo das possíveis configurações de quadrotores e suas classificações. Foi feito um estudo dos principais componentes e opções disponíveis no mercado para integrar essa plataforma. E por último o estudo das principais funcionalidades envolvidas nos projetos de quadrotor, como localização, controle, planejamento de trajetória e energia. 


\section{Considerações finais}
\label{sec:consid}

A pesquisa trouxe conhecimento significativo para a tomada de decisão em relação a escolha entre os designs possíveis para o desenvolvimento de uma plataforma do tipo quadrotor se mostrando a configuração em "x" a que traz mais benefícios. Escolha da estratégia de controle a ser utilizada, escolha entre os planejadores de trajetória e principais componentes envovlvidos no projeto.


