\chapter{Estudo do Estado da Arte}
\label{chap:metod}
Nessa pesquisa foram abordados diversos aspectos que envolvem a concepção de um projeto envolvelmento um UAV do tipo quadrotor. O desenvolvimento de uma plataforma desse tipo ainda envolve desafios estruturais, autonomia, controle, localização entre outros, que necessitam de estudo prévio detalhado para ser alcançado um bom resultado com o veículo.

% %--------- NEW SECTION ----------------------
\section{Quadrotores}
% \label{sec:ui}
Quadrotores são aeronaves de asas rotativas, ou seja, são sustentandas e movimentadas por rotores. Diferente das aeronaves de asas fixas, como aviões, os aeronaves de asas rotativas não utilizam seu movimento horizontal para sustentar seu vôo. Isso faz com que esse tipo de veículo apresente um consumo energético muito maior. Apesar disso, as aeronaves de asas rotativas possuem a habilidade de realizar pouso e decolagem vertical, além de poder realizar vôos estacionários.

As aeronaves de asas rotativas são classificadas como multirotores, sendo seus tipos mais importantes quadrotores, hexarotores, octarotores, coaxiais ou helicópteros. Os quadrotores são veículos que apresentam alta manobrabilidade e alto payload, mas apresentam também alto gasto energético, tornando baixa o seu tempo de vôo, sendo um desafio achar baterias mais efieciente que aumentem sua autonomia. A medida que aumenta-se o número de rotores, passando de quadrotores para hexarotores e octarotores, aumenta-se também o payload, que é o quanto a aeronave consegue carregar em relação ao seu peso, e a tolerância a falha, que é continuar realizando um vôo controlado mesmo com algumas das hélices apresentando falhas de funcionamento. Entretanto, diminui-se também a manobrabilidade e aumenta-se o consumo energético.

\subsection{Classificações}
Os quadrotores são classificados em diversas categorias, possuindo cada uma delas caracacterísticas específicas, que podem ajudar no design do projeto e também na escolha de componentes que vão ajudar na operação da aeronave.

\subsubsection{Classificação Quanto ao Peso}
Os UAVs são classificados em diversas categorias de acordo com o quanto eles pesam. UAVs que pesam até 2 quilogramas (kg) são classificados como micro, de 2 kg até 7 kg são classificados como mini, de 7 kg até 25 kg são classificados como pequenos, de 25 kg até 150 kg, como médios e de 150 kg em diante são classificados como grandes.\\
Quadrotores mais leves são mais ágeis por serem menores e por isso terem inércias menores. Possuindo assim maiores. acelerações angulares 

\subsubsection{Classificação Quanto a Configuração}
Os quadrotores tem duas configurações possíveis em relação à disposição de seus rotores. Os rotores podem ter a a configração em forma de "+", em que a frente da aeronave fica alinhada com uma das hastes que suporta um par de rotores. Essa configuração também é conhecida como cruz. A outra confuração possível é a configuração em "x", em que a frente da aeronave fica a 45$^{\circ}$ do eixo que contém a haste da aeronave, ficando assim a frente da aeronave no meio de duas hastes, como mostrado na Figura.

A configuração em "+" é mais acrobática, entretando, como desvantagens, a haste dos rotores bloqueia o campo de visão da câmera e também apresenta um momento de guinada ao translacionar, necessitando de um maior gasto energético para estabilizar a aeronave.

A configuração em "x" não apresenta esse efeito, por ter os movimentos de arfagem e rolagem desacoplados do de guinada. Apresenta menor esforço para transladar pois todos rotores agem nesses movimentos, diferente da configuração em "+", em que apenas um par de rotores é responsável pelo deslocamento enquanto o outro se mantém com velocidades constantes. É mais estável, entrentando apresenta menor manobrabilidade.

\subsubsection{Ambiente de Operação}
Os quadrotores podem operar em dois tipos de missões: outdoor e indoor. As missões outdoor são as missões em que os quadrotores são expostos a ambientes desconhecidos, onde existe a forte presença de pertubações, como rajadas de vento. Nesse tipo de missão, os quadrotores muitas vezes vão precisar de sensores do tipo GPS para ajudar na localização do veículo e também de um controlador adequado para lidar com a rejeição de pertubação. As missões indoor possuem menos pertubações e ambientes mias estruturados. Sendo possível fazer o mapeamento prévio do ambiente para realizar as operações 

\subsection{Principais Componentes}

\subsubsection{IMU}

\subsubsection{Rotores}

\subsubsection{Baterias}

\subsubsection{Microprocessador}

\section{Revisão Bibliográfica}
% \label{sec:ui}
\subsection{Autores}
\subsection{Artigos}

\section{Ambiente de Aplicação}
% \label{sec:ui}
O uso do veículos aéreos não tripulados do tipo quadrotor tem se expandido em diversas áreas, tanto civíl, quanto militar e também acadêmica.

No ambiente acadêmico, os quadrotores são utilziados como plataforma para teste de estratégias de controle, dada a dificuldade de se estabilizar e de controlar esse tipo de veículo, e também para teste de técnicas de planejamento de trajetória, por se deslocar no espaço tridimensional.

Os quadrotores por não necessitarem de um piloto embarcado, poderem ser pequenos e ágeis podem ser utilizados em missões de busca e resgate em abientes hostis, como casos de desmoronamento ou soterramento, e também em missões de espionagem e vigilância, por chamaerem pouca atenção.

Tem-se crescido muito o uso de aeronaves do tipo quadrotor na área de agricultura de precisão. Os drones podem ser utilizados para monitorar uma lavroura, mapear, distribuir uso de defensivos, semear e até mesmo irrigar.

Outra área de grande utilização é a de enterterimento. Existem muitos drones são comercializados com finalidade exclusivamente lúdica, podendo ser equipados com câmeras para produzir fotografias e vídeos. São utilizados tambpem na cinematografia para realizar essas filmagnes aéreas, que antes eram feitas por helicópteros tripulados que tinham custo associados muito maiores.


\section{Funcionalidades}
% \label{sec:ui}

\subsection{Modelagem}
A modelagem de um quadrotor é uma das etapas mais importantes no desenvolvimento de um projeto envolvendo esse tipo de veículo. Por ser uma plataforma instável, se torna inviável realizar técnicas de identificação em malha aberta. Sendo assim, é necessário obter o modelo dinâmico da aeronave através de técnicas de modelagem.

A modelagem da aeronave pode ser obtida através das equações de Newton-Euler ou através do formalismo de Euler-Lagrange (CASTILLO 2005). Através dessa modelagem é possível obter o modelo de alto nível, onde os torques e forças são entradas e as saídas são posições angulares e lineares. 

\subsection{Controle}
Os quadrotores são veículos subatuadas, ou seja, possuem mais graus de liberdade do que atuadores, são naturalmente instáveis e apresentam comportamento não-linear. Devido a esses fatores, esse tipo de veículo necessita de uma estrutura de controle adequada bem ajustada para que sejá possível a estabilização e o seguimento de referência.

Os controladores que atuam no quadrotor geralmente são utilizados em cascata (NONAMI,2010), de forma que existe um controle de baixo nível para garantir uma velocidade de rotação desejada nos rotores, um controlador em um npivel mais alto para controlar a altitude e as velocidade angulares de rolagem, arfagem e rolagem e por último um controlador no nível mais alto controladando posições lineares no espaço tridimensional. (kendoul)

Controlador hierárquico, estabilidade global

Os controlador mais comumente usado em baixo nível para controlar a velocidade de de roatação dos propulsores é o PID. 

Os controladores que controlam atitude, altitude e posições lineares no quadrotor podem ser controladores lineares ou não-lineares. Para serem utilizados controladores lineares é necessário realizar uma linearização do modelo matemático do quadrotor, considerando pequenas variações de ângulo. Os controladores lineares que são mais amplamente utilizados são os controladores PID, LQR, H2, H$\infty$ e controle adaptativo. Como alguns parãmetros do quadrotor podem apresentar incertezas ou até mesmo variar durante a operação, também é possível adcionar robustez a esses controladores, podendo ajudar também na rejeição de pertubação.

Os controladores não-lineares mais amplamente utilizados são os controladores por inversão de modelo, o sliding mode control (SMC) e o controlador Fuzzy.

\subsection{Localização}
A localização do quadrotor pode ser auxiliada por uso de diversos sensores como LiDAR, GPS, IMU, câmeras monoculares, sensores ultrassônicos e lasers.

A IMU pode ser utilizada para a funcionalidade de localização através da odometria. A IMU pode contar com giroscópio, acelerômetro, magnetômetro e até mesmo barômetro.

Os sensores que são mais amplamente utilizados atualmente são os sensores baseados em visão. É possível obter bons resultados apenas utilizando câmeras monoculares, como mostrado em ... utilizando o pacote ORB Slam.

Os sensores GPS são amplamente utilizados para missões outdoor e os sensores ultrassônicos e baseados em laser são utilizados para auxiliar na medição da altitude.

Pode-se realizar a fusão sensorial de diversos sensores para se obter uma boa estimativa da localização da aeronave. A técnica mais amplamente utilizada é a de filtragem, como a do filtro extendido de kalman, EKF, porém ela sofre com o drift, que é um deslocamento não considerado pela medição. Outra opção é o ? framework de otimização não-linear, que apresenta resultados mais consistentes, porém apresenta um custo computacional superior.

\subsection{Planejamento de Trajetória}
Existem diversos algoritmos que podem ser utilizados para realizar o planejamento de trajetória do quadrotor.
\begin{itemize}
    \item Artificial potential field
    \item probabilistic roadmap
    \item belief roadmap
    \item potential fields (PF)
    \item Heuristic search algorithms
    \item Optimization methods
    \item planning under uncertainties
    \item reactive and bio-inspired obstacle avoidance methods
\end{itemize}


\section{Mapa Conceitual}
% \label{sec:ui}